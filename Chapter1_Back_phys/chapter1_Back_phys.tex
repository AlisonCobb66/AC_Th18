%*******************************************************************************
%*********************************** First Chapter *****************************
%*******************************************************************************

\graphicspath{{Chapter1_Back_phys/Figs/}}

\chapter{Background physics}  %Title of the First Chapter

Further details of the structure and behaviour of tropical and extra-tropical cyclones are now presented in sections \ref{TC} and \ref{ETC}, respectively.


\section{Tropical cyclones} \label{TC}

% Formation of TCs
% Tropical storm structure and energy lifecycle
% Influence of the ocean
% Influence of the atmosphere
% Tropical cyclone variability in the West Pacific
% Long-term variability
% Observations
% Models

Globally, there are approximately 80-100 tropical storms each year. The greatest frequency is observed in the Western North Pacific (WNP), with an average of 26 per year since 1981 \citep{zhan2012seasonal}. From 1977-2011, China had the highest rate of landfalls globally, averaging 6.4 per year \citep{HironoriFudeyasu:178}. Not only does the West Pacific experience the greatest number of tropical storms and landfalls per year, but also the most intense and largest. Tropical storms can occur throughout the year in the West Pacific, but the most active season is June-November.

A tropical storm is a low pressure system that forms over tropical or sub-tropical waters (i.e. 23.4$^{\circ}$N to  23.4$^{\circ}$S), with organised convection and winds at low levels circulating anti-clockwise in the northern hemisphere or clockwise in the southern hemisphere. 

\begin{figure}[h]
	\centering
	\noindent\includegraphics[width=15pc,angle=0]{typhoonsizes.jpg}
	\caption{Comparison of tropical cyclone sizes: Super Typhoon Tip and Tropical Cyclone Tracy. Source: \cite{noaa_structure}}\label{fig:cyclone_size}
\end{figure}

They are hugely variable in terms of size, illustrated in figure \ref{fig:cyclone_size}, where Tropical Cyclone Tracy (1974) covers just 0.03\% the area of Super Typhoon Tip (1979). Tropical cyclones tend to extend throughout the depth of the troposphere, approximately 16 km. The Saffir-Simpson Hurricane Scale \citep{simpson1974hurricane} is often used to categorise tropical storms by intensity based on maximum winds. Storms with winds of 38 mph (61 km/h, 33 knots) or less are called 'tropical depressions' and above this they are called 'tropical storms'. With increasing intensity there are category 1 to 5 'hurricanes', with categories 3,4,5 referred to as 'major hurricanes'. In the West Pacific basin, if maximum sustained winds reach 64 knots (33 m/s, 74 mph) the term 'typhoon' is used, and a 'Super Typhoon' is if the maximum sustained winds are at least 130 knots (67 m/s, 150 mph). The speed they move along the underlying surface, or 'translation speed' is around 10-15mph (10 knots), but can slower or as fast as 40 mph \citep{mo_tc}.  

A recent notable tropical storm in the Western North Pacific is Super Typhoon Haiyan (2013). This is one of the most powerful typhoons ever to make landfall, with maximum sustained winds of 170 knots (88 m/s, 195 mph). Haiyan tracked over the Philippines archipelago, with storm surge primarily responsible for 6,300 dead, 1061 missing and almost 30,000 injured \citep{lagmay2015devastating}. %Average 472 fatalities a year (1983-2006) \citep{zhang2009tropical} 

 %(http://www.srh.noaa.gov/jetstream/tropics/tc\_classification.html)
%(I need to remember if 1-min or 10-min. Which is US? Which is WMO?)

%Size is not necessarily an indication of hurricane intensity. 

%rotating winds of a tropical cyclone, combined with the north–south variation in the Coriolis parameter, induce relative vorticity asymmetries in the tropical cyclone (Fig. 8.60). These asymmetries are called the β–gyres.
%Tropical cyclones move in relation to the integrated, deep-layer (to \textasciitilde 17 km) environment flow in which they are embedded \citep{neumann1985role}.
% annulus degrees = see Carr 
%The motion of a tropical cyclone is highly correlated with latitude \citep{neumann1985role}. 
%5Chan 1985. (ref atm and ocean control sections). Low latitude easterlies and high lat westerlies, subtropical high.


\subsection{Formation of tropical cyclones}

There are a number of environmental factors that need to be satisfied in order for a tropical storm to generate at a location. These are:
\begin{itemize}
	\item Presence of a convective system
	\item Non-negligible Coriolis force (at least 500 km, 300 miles) from the equator) \citep{noaaA15}
	\item Low wind shear (less than about 20 knots (10 m/s, 23 mph) \citep{noaaA15} 
	%	\item Enhanced vorticity
	\item Sufficient humidity in the mid-troposphere
	\item Warm ocean water of at least 26.5$^{\circ}$C throughout a sufficient depth (50m)
	
\end{itemize}

Tropical cyclones cannot be generated spontaneously. To develop, they require a weakly organized system with sizeable spin and low level inflow. For tropical cyclogenesis to occur, there is a requirement for the Coriolis force to provide for near gradient wind balance to occur. Without the Coriolis force, the low pressure of the disturbance cannot be maintained. Large values of vertical wind shear disrupt the incipient tropical cyclone and can prevent genesis, therefore little wind shear is important. Relatively moist layers near the mid-troposphere (5 km, 3 miles). Dry mid levels are not conducive for allowing the continuing development of widespread thunderstorm activity. The development and maintenance of tropical cyclones require a warm ocean surface to act as a source of energy.

%Tory = enhanced cyclonic low-level vorticity .A pre-existing near-surface disturbance with sufficient vorticity and convergence. 
%Tropical waves spawn tcs... AEW, MT
%An atmosphere which cools fast enough with height such that it is potentially unstable to moist convection. 
%Tory - Tory - deep convection


\subsection{Tropical storm structure}

The main parts of a tropical cyclone are the dense cirrus overcast, rainbands, the eye, and the eyewall (figure \ref{fig:cyclone_structure}). There is boundary layer inflow in a cyclonic direction (anti-clockwise in the northern hemisphere) and anti-cyclonic (clockwise) upper tropospheric outflow.


\begin{figure}[h]
	\centering
	\noindent\includegraphics[width=20pc,angle=0]{hurr_cross.jpg}
	\caption{Tropical cyclone structure. Source: \cite{noaa_structure}}\label{fig:cyclone_structure}
\end{figure}

\subsubsection{Eye and eye wall}
The circular 'eye' or centre of a tropical cyclone is an area of slowly sinking air, characterised by light winds (usually do not exceed 15 mph (24 km/h)) \citep{noaa_structure} and little or no precipitation (figure \ref{fig:cyclone_structure}). Eye diameters are typically 40 km but can range from under 10 km to over 100 km \citep{bom_tc}. Although the winds are calm at the axis of rotation, strong winds may extend well into the eye. The eye is the region of lowest surface pressure and warmest temperatures aloft.

\begin{figure}[h]
	\centering
	\noindent\includegraphics[width=14pc,angle=0]{warm_core2.png}
	\caption{Hurricane Bonnie temperature cross-section: the warm core. Source: \cite{eastin}}\label{fig:warm_core}
\end{figure}

Figure \ref{fig:warm_core} shows the maximum temperature anomalies present in the upper levels of the eye of Hurricane Bonnie. This is a result of subsidence and adiabatic heating in the eye, and eyewall latent heat release. The warm core is responsible for the extremely low surface pressures in the eye and large pressure gradients across the eyewall. The eye temperature may be 10$^{\circ}$C warmer or more at an altitude of 12 km (8 miles) than the surrounding environment, but only 0-2$^{\circ}$C warmer at the surface (Hawkins and Rubsam 1968) in the tropical cyclone (figure \ref{fig:warm_core}).

The eye is surrounded by a dense ring of cloud deep convection about 16 km high known as the eye wall which marks the belt of strongest winds and heaviest rainfall (\citep{bom_tc}). Changes in the structure of the eye and eyewall can cause affect the wind speed within the storm. The eye can grow or shrink in size, and double (concentric) eyewalls can form. In intense tropical cyclones, some of the outer rainbands may organize into an outer ring of thunderstorms that slowly moves inward. The inner eye wall (and storm intensity) weakens as it feels the effects of the subsidence resulting from this outer eyewall, inhibiting the inner eyewall of its needed moisture and momentum. Eventually the outer eyewall replaces the inner one completely and the storm can be the same intensity as it was previously or, in some cases, even stronger.% REF this section about weakening and intensifying. The pressure rises due to the destruction of the inner eyewall are usually more rapid than the pressure falls due to the intensification of the outer eyewall, and the cyclone itself weakens for a short period of time. 
%Some of the most intense tropical cyclones exhibit concentric eyewalls, two or more eyewall structures centered at the circulation center of the storm ( Willoughby et al. 1982,Willoughby 1990a ). Just as the inner eyewall forms, convection surrounding the eyewall can become organized into distinct rings. 

There remains some debate about the mechanism by which the eye forms \citep{noaa_a11}. It may be due to the downward directed pressure gradient associated with the weakening and radial spreading of the tangential wind field with height (Smith, 1980), or subsidence forced by latent heat release in the eye wall (Shapiro and Willoughby, 1982), or a combination of mechanisms.

%The exact mechanism by which the eye forms remains somewhat controversial. One idea suggests that the eye forms as a result of the . Another hypothesis suggests that the eye is formed when latent heat release in the eyewall occurs, forcing subsidence in the storm's center . It is possible that these hypotheses are not inconsistent with one another. In either case, as the air subsides, it is compressed and warms relative to air at the same level outside the eye and thereby becomes locally buoyant. This upward buoyancy approximately balances the downward directed pressure gradient so that the actual subsidence is produced by a small residual force.  \citep{noaa_a11}

%At around 74 mph (119 km/h) the strong rotation of air around the cyclone balances inflow to the center, causing air to ascend about 10-20 miles (16-32 km) from the center forming the eyewall. This strong rotation also creates a vacuum of air at the center, causing some of the air flowing out the top of the eyewall to turn inward and sink to replace the loss of air mass near the center.

\subsubsection{Rain bands}
Convection in tropical cyclones is organized into long, narrow rainbands which are oriented in the same direction as the horizontal wind (figure \ref{fig:cyclone_structure}). Along these bands, low-level convergence and upper-level divergence are at a maximum. Warm, moist air converges at the surface, ascends through these bands, diverges aloft, and descends on both sides of the bands. Subsidence is distributed over a wide area on the outside of the rainband but is concentrated in the small inside area \citep{noaa_a11}. As the air subsides, adiabatic warming takes place, and the air dries. Because subsidence is concentrated on the inside of the band, the adiabatic warming is stronger inward from the band causing a sharp contrast in pressure falls across the band since warm air is lighter than cold air. Because of the pressure falls on the inside, the tangential winds around the tropical cyclone increase due to increased pressure gradient and eventually the band moves towards the centre (Willoughby 1979, 1990a, 1995). 

% Eventually, the band moves toward the center and encircles it and the eye and eyewall form 

%What is the intensity of rain in a TC?


%\subsection{Tropical cyclone energy and lifecycle}

Within a tropical cyclone, there are two distinct circulations referred to as 'primary' and 'secondary' (figure \ref{fig:cyclone_circ}). The primary circulation is what visibly characterises the phenomenon, with winds swirling cyclonically around the cyclone eye. The secondary circulation is the flow of air into the centre, ascending in the eye and then divergence at upper levels.

%\begin{figure}[h]
%	\centering
%	\noindent\includegraphics[width=20pc,angle=0]{H:/Documents/Admin/ESA/gradient_wind_balance.png}
%	\caption{Gradient wind balance in primary circulation of a tropical cyclone. Source: \cite{circ_pic}}\label{fig:cyclone_circ}
%\end{figure}
%% Do not cite circ_pic as this reference url has '%', which means all following references are missed

\subsubsection{Gradient wind balance}% and thermal wind balance}

The basic horizontal balance in a tropical cyclone above the boundary layer is between the sum of the Coriolis and centripetal forces, balanced by the horizontal pressure gradient force. This balance is referred to as gradient balance (figure \ref{fig:cyclone_circ}). %The centripetal force alters the original two-force geostrophic balance and creates a non-geostrophic gradient wind.
The classic theory for hurricane intensification relies on the inflow induced by the deep convection. However, as the vortex strengthens, the boundary layer becomes increasingly important \citep{under_hurr}.

%A symmetric tropical cyclone is in approximate thermal wind balance \citep{comet_tm}. The thermal wind is the difference between the geostrophic wind at two vertical levels and therefore represents the vertical wind shear of the geostrophic wind.

%what goes on in the boundary layer?
%http://www.goes-r.gov/users/comet/tropical/textbook_2nd_edition/navmenu.php_tab_9_page_2.4.1.htm

%The reason that different peak winds can result in different central pressures is caused by the fact that the radius, r, of the peak wind varies. A storm with 40 m/s peak winds with a 100 km RMW will have a much lower pressure drop than one with a 25 km RMW.

\begin{figure}[h]
	\centering
	\subfloat[1a]{\includegraphics[width=2.4in]{gradient_wind2006.png}}
	\subfloat[2a]{\includegraphics[width=2.4in]{gradient_wind2006B.png}}
	
	%	\noindent\includegraphics[width=20pc,angle=0]{H:/Documents/Admin/ESA/gradient_wind2006.png}
	\caption{a) gradient wind force balance in the primary circulation of a tropical cyclone b) disruption of gradient wind balance by friction in the boundary layer leaving a net inward pressure gradient that drives the secondary circulation with inflow in the boundary layer and outflow above it. Source: Smith, 2006}\label{fig:cyclone_circ}
\end{figure}

Surface friction in the boundary layer reduces the wind speed near the surface and therefore the centrifugal and Coriolis forces, but has little effect on the pressure field. There is therefore a new inward force in the boundary layer, driving inflow and the secondary circulation. The conservation of angular momentum means is objects will spin faster as they move toward the centre of circulation, so air increases its speed as it heads toward the centre of the tropical cyclone.


\subsubsection{Potential intensity}

To a good approximation, the secondary circulation in a tropical cyclone is a natural realisation of a Carnot heat engine, except that the engine does no work on its environment, the available work is locally dissipated and a fraction of the dissipated energy is recycled into the engine. A Carnot heat engine is the most efficient heat engine cycle allowed by physical laws. The Carnot perspective provides an upper bound on the maximum wind speed that a storm can attain.

\begin{figure}[h]
	\centering
	\noindent\includegraphics[width=20pc,angle=0]{carnot1.jpg}
	\caption{The Carnot cycle in a tropical cyclone. A-B: isothermal inflow of near-surface air; B-C: moist adiabatic ascent in the eye wall and outflow just below the tropopause; C-D: sinking of cooled air in the environment far from the tropical cyclone center. To close the system, D-A: the cooled air is assumed to return to the tropical cyclone environment adiabatically. Source:\cite{goescarnot}}\label{fig:cyclone_carnot}
\end{figure}

%CISK / WISHE
%\cite{emanuel1991theory} ref for carnot cycle
%vortical hot tower

%As this air ascends, 90\% of the stored energy is released by condensation, giving rise to the towering cumulus clouds and rain. The release of heat energy warms the air locally, causing a further decrease in pressure aloft. Consequently, air rises faster to fill this area of low pressure, and more warm, moist air is drawn off the sea, feeding further energy to the system. Thus, a self-sustaining heat engine is created.
%As little as 3\% of the heat energy may be converted into mechanical energy of the circulating winds. This relatively small amount of mechanical energy equates to a power supply of 1.5x1012 Watts - equivalent to about half the world-wide electrical generating capacity!
%http://www.metoffice.gov.uk/weather/tropicalcyclone/facts
%Emanuel theory 
%static vs dynamic theory

%In the northern hemisphere, positive vorticity at low levels and negative at upper levels. Vorticity decays with height. 

%It is the thunderstorm activity which allows the heat stored in the ocean waters to be liberated for the tropical cyclone development. (deep surface layer of conditional instability) - see Tory

%Intensification and decay
%Large vertical shear can weaken or destroy the tropical cyclone by interfering with the organization of deep convection around the cyclone center.
%(noaa website)
%maturity

%gray, frank montgomery, smith, tory book chapter- number of environmental conditions require to be satisfied for tc genesis.
%Emanuel
%ventilation
%Warm core is in thermal wind balance with the primary circulation.

A tropical cyclone is approximately 500-1000 km wide and 10-16 km high, set in much larger scale flow, so it can be treated as a solid object floating in the atmosphere \citep{chan2005physics}. Environmental steering is typically defined as the wind within an annulus centred on the tropical cyclone. e.g. 5-7 degrees (\citep{chan1982tropical}, \citep{chan1985identification} or 3 degrees (Franklin 1996). To define the steering flow, tropospheric layer means are better than single-level analyses \citep{velden1991basic}, with the deep layer mean (DLM)(1000-100 hPa) or mid-tropospheric level (500-700 hPa) often used. The vector quantity for the difference between the environmental steering and tropical cyclone motion is termed 'propagation' \citep{carr1990observational}. This difference arises from variations in the Coriolis parameter and environmental vorticity across the tropical cyclone. 

Depending on direction of movement of the storm, there is a different relationship with the surrounding flow \citep{chan1985identification}. Westward moving TCs move faster and to the left of the steering in the Northern Hemisphere, and eastward moving TCs move slower and to the right \citep{carr1990observational}.
%Importance of zonal direction of cyclone translation in determining the relationship between the environmental flow and cyclone movement. 

%Steered by surrounding flow and modified by Coriolis force (beta effect) and the horizontal vorticity gradient of the surrounding flow (chan physics review).
%NH Tcs move 10-20 degrees to the left and faster by 1ms than mid-tropospheric (700, 600, 500) at 6 degrees. 
%The beta effect - the cyclone and environment interact to modify the basic flow, and the vortex is then steered by this modified flow.

Studies have shown that depth of the steering layer is related to the strength of the cyclone \citep{velden1991basic}. For a more intense storm, there is greater vertical development of the cyclonic vortex, which is then advected by an environmental flow of greater depth. 

Tropical cyclone translation after landfall is affected by terrain, circulation environment, steering flow, TC intensity and structure among others \citep{xiao2013analysis}.
% effects of land, e.g. Philippines, Taiwan. track variation - interaction with topography (Wu and Kuo 1999), Taiwan


%Previous investigation into the sudden change in tropical cyclone track in four storms in the WNP suggested that such changes occurred near the centre of the MJO-scale cyclonic circulation or at the birfurcation of steering flows at 700 hPa \citep{wu2011observational}.

%Tropical cyclones forming between 5 and 30 degrees North latitude typically move toward the west. Sometimes the winds in the middle and upper levels of the atmosphere change and steer the cyclone toward the north and northwest. When tropical cyclones reach latitudes near 30 degrees North, they often move northeast

%Kim et al MJO effect
%Monsoon trough effect


%%%%%%%%%%%%%%%%%%%%%%%%%%%%%%%%%%%%%%%%%%%%%%%%%%%%%%%%%%%%%%%%%%%%%%%%%%%%%%%%%%%%%%%%%%%%%%%%%%%%%%%%

\subsection{Influence of the ocean on tropical cyclones} \label{ocean}

The ocean provides a source of energy to the tropical cyclone system by air-sea sensible and latent heat fluxes, determined by the sea surface temperature (SST). Many studies have shown that the SST must exceed 26$^{\circ}$C for tropical cyclones to form \citep[e.g.][]{palmen1948formation}. However, there is much research into this threshold \citep[e.g.][]{dare2011threshold, mctaggart2015revisiting}, highlighting basin-dependence and the importance of ocean temperature below the sea surface. The WNP has high SST throughout the year (\textgreater 28.5$^{\circ}$C) \citep{chan2007interannual}, relative to the suggested 26.5$^{\circ}$C threshold for genesis (section 1.2.1). Variability in activity is related to the SST pattern, with warm anomalies favouring intensification. The West Pacific is warm, past threshold limit, threshold more suitable for the North Atlantic, where more marginal.

The potential intensity of a tropical cyclone is directly related to the SST below the cyclone, all else being equal \citep{emanuel1991theory, holland1997maximum}, with higher SST promoting increased intensity. At high wind speeds, the surface wind stress generates strong turbulent mixing within the ocean. This causes entrainment of cooler water towards the surface from the thermocline and deepens the mixed layer.  The cooling of the SST is determined by the initial state of ocean, storm intensity, translation speed and storm size, and limits storm intensification, especially for slow moving storms, where SST cools more \citep{bender1993numerical, bender2000real}. During the lifetime of a storm, the wake produced can change considerably, for example Megi (2010) created a wake of 1.6$^{\circ}$C whilst in the Philippine Sea, but once in the South China Sea, cooling increased greatly to 7$^{\circ}$C \citep{d2014impact}. Due to this turbulent entrainment of cold water into the oceanic mixed layer induced by the TC, the state of ocean below the surface is also important to the cyclone system \citep{bender2000real, shay2000effects}.
%(see if any emanuel refs here - environmental control on...)
%Price 1981 - cooling 1 to 6C. \citep{price1981upper}
%Using satellites, observations of this cooling has been possible, 
%Cold wake of ~6C \citep{prasad2007upper}. 
%check Bender2000real.
%SST decrease induced by passage of a TC is approximately 1-6$^{\circ}$C  \citep{price1981upper}. 

The ocean heat content (OHC) in Joules (J) is a measure of the heat content within the ocean between two reference levels. 

\begin{equation}
OHC = c_{p} \int_{z1}^{z0} (T-T_{ref}) \rho dz
\end{equation} 	
%https://www.sharelatex.com/learn/Mathematical_expressions

%\begin{equation}
%$\displaystyle OHC = c_{p} \int_{z1}^{z0} (T-T_{ref}) \rho dz$
%\end{equation} 	

Tropical cyclone heat potential (TCHP) is vertically integrated heat content from the sea surface down to the 26$^0$C isotherm, which as at depth 'D26'.
% cannot find ref: as Gray (1968, 1978) suggested that depth of 60m is necessary for intensification.
%When this heat content is from the surface down to the 26$^0$C isotherm, it is termed tropical cyclone heat potential %(TCHP).
%Increased availability of sub-surface ocean observations, e.g. ARGO floats, since ...
%The ocean is the source of energy for a TC's intensification, typically down to 100-200m is important (Emanuel 1999) \citep{bender2000real}, \citep{shay2000effects}.
%Although WPAC SST is warm (generally above xyz), anomalously warm water can favour intensification. 

If the ocean warm layer (D26) is relatively shallow, (e.g. 60m) a positive SST anomaly is critical to intensification because the features can effectively deepen the warm layer and restrict the TC-induced cooling. If this layer is deep, e.g.\textgreater 105m, a warm feature is not required as the background is already sufficient to overcome the negative cooling feedback \citep{lin2008upper}.
%But isnt the warm SST anomaly needed in the first place?

In the WNP, intensification to category 5 needs SST around 29$^0$C and subsurface heat content required depends on the storm translation speed \citep{lin2009upper}. A shallow warm subsurface (D26 \textasciitilde 60m) is sufficient to intensify to a category 5 in a fast moving storm, but for slow moving storms, a much deeper warm layer is required \citep{lin2009upper}.

%Rapidly moving storms with deep oceanic mixed layer, SST feedback is of minor importance \citep{schade1999ocean}. All hurricanes attain their maximum intensity over warm ocean waters (does this mean the warmest or just above a threshold??) - from Prasad, referencing Goni and Trinanes, 2003

Typhoon Imbudo (2003) intensified from 56 knots (29 m/s, 65 mph) to 113 knots (58 m/s, 130 mph) in 12 hours when it passed over a region that increased TCHP by 100 kJ/cm$^2$. As the typhoon passed over these waters, the SST decreased by 3-4$^0$C alongside TCHP, but there was sufficient energy to negate the negative feedback of upwelling of cooler waters \citep{goni2003ocean}.
%But still there was cooling?


%On a much smaller scale, breaking waves and sea spray produced by TCs may change the wind stress \citep{moon2004effect}. Surface waves are a source of surface friction, moisture flux and ocean mixing and are a non-linear function of fetch and wind speed.

%Breivik et al. (2015) recently have demonstrated substantial improvement in climatological ocean biases with explicit treatment of surface waves. Waves generated by tropical cyclones are an important cause of infrastructure damage so an explicit and coupled treatment is desirable.

%%%%%%%%%%%%%%%%%%%%%%%%%%%%%%%%%%%%%%%%%%%%%%%%%%%%%%%%%%%%%%%%%%%%%%%


\subsection{Influence of the atmosphere on tropical cyclones}
%The atmosphere can maintain the TC or destroy it, principally through vertical wind shear.
A cyclone exists throughout the depth of the troposphere and so the atmospheric conditions have a large effect on the phenomenon. It has been established that vertical wind shear is detrimental to tropical cyclone genesis and intensification \citep[e.g.][]{chan1982tropical, McBride1995}, although mature, large tropical storms may resist relatively large wind shear \citep{zeng2007environmental}.

%One hypothesized pathway by which vertical shear affects tropical cyclones is mid-level ventilation, or the flux of low-entropy air into the centre of the tropical cyclone \citep{McBride1995}. Tropical upper tropospheric trough (TUTT) cells from the mid-latitudes can weaken tropical cyclones by introduction of vertical wind shear, but also bring a cyclonic PV anomaly, which may contribute to intensification \citep{zeng2007environmental}.
%EXPLAIN
%Strong vertical wind shear prohibits rapid intensification and most likely results in the weakening of TCs \citep{zeng2007environmental}.
%and a threshold of 12.5 ms$^-1$ above which TCs cannot form in the WPAC has been suggested \citep{zehr1992tropical}-seems to be a thesis - check this ref
%e and Zehr 1991, Zehr 1992 - thesis?)
%check zeng paper
% McBride and Tang for ventilation

The translation of TCs is largely determined by the large-scale atmospheric pattern, with mid-tropospheric levels (500 and 700 hPa) found to have the best correlation with TC direction and speed \citep{chan1982tropical} (\ref{steer}). If the passage of the storm is too slow, the cooling induced by the storm will inhibit intensification, and if it is moving too fast, the resulting asymmetric structure will also inhibit intensification \citep{zeng2007environmental}. Rapid intensification (RI) is An increase in the maximum sustained winds of a tropical cyclone of at least 30 kt in a 24-h period (\citep{nhc_gloss}). and for this to occur, a translation speed between 3-8 m/s is required. % (REF). % ref missing here

Although wind is the primary atmospheric control on TC activity, relative humidity and vorticity are also important.

% Fast translation speed and strong vertical shear and detrimental to TC intensification \citep{zeng2007environmental} and

%%%%%%%%%%%%%%%%%%%%%%%%%%%%%%%%%%%%%%%%%%%%%%%%%%%%%%%%%%%%%%%%%%%%%%%
\subsection{Tropical cyclone variability in the West Pacific}

Changes in atmospheric and oceanic conditions drive tropical cyclone variability. Tropical cyclone activity in most ocean basins including the WNP has a strong interannual signal \citep{landsea2000climate}, with variability observed in genesis location, track, intensity, landfall and lifetime.% (could use zhan2011contributions).

%Regions preferable for genesis. Or something like, genesis location is important and atmospheric conditions to maintain or dissipate storms are important. Distribution of SST important, also ocean at depth is important.

%(At the end mention interdecadal seasonal variability)

\subsubsection{El Ni\~{n}o}

It has been long established that the El Ni\~{n}o-Southern Oscillation (ENSO) (figure \ref{fig:nino}) is the principal driver of interannual variability in the Tropics. This phenomenon also has a marked effect on tropical storm activity in all basins. In a warm El Ni\~{n}o year, when the SST in the central and eastern equatorial Pacific is anomalously warm, there is a zonal displacement of annual mean tropical storm genesis location eastward \citep{lander1994exploratory, zhan2011contributions}. These storms tend to have a longer lifetime and can reach higher intensities before they recurve or meet land \citep{camargo2007cluster, chan1998seasonal}. In contrast, during La Ni\~{n}a years, the main region of genesis shifts westwards, with fewer intense storms.
%El Nino -weakening of Walker circulation. La Nina - strengthening.
%, Chan 2000, Wang and Chan 2002, for zonal displacement ENSO.


\begin{figure}[h]
	\centering
	\noindent\includegraphics[width=40pc,angle=0]{Stressors_ENSO3.png}
	\caption{El Ni\~{n}o, normal and La Ni\~{n}a conditions in the atmosphere and ocean in the tropical Pacific. Source: \cite{noaa_enso}}\label{fig:nino}
\end{figure}


There is a significant difference in landfall location between El Ni\~{n}o and La Ni\~{n}a years. During El Ni\~{n}o, with more storms generated in the southeast quadrant of the WNP, they tend to recurve before landfall and affect Japan and Korea, with fewer across the Philippines and South China Sea \citep{liu2008interdecadal}. Whereas in La Ni\~{n}a years, the storms generate further westwards and are straight moving, with increased landfalls observed in China \citep{camargo2007cluster}.
%Does that mean that cyclones hitting China are weaker?

%ENSO affects the distribution of tropical storm numbers within the season \citep{lander1994exploratory}, as well as the landfall statistics, with \cite{yonekura2011statistical} finding significantly higher landfall rates in all coastal regions in La Ni\~{n}a.

%%%% MAKE FIGURE %%%%
% 	\begin{figure}[h]
% 		\noindent\includegraphics[width=20pc,angle=0]{Y:/Code_Data/Plots_NCAR/MAM/Maps/regions.png}
% 		\caption{Tropical storm tracks in 5 El Ni\~{n}o years and 5 La Ni\~{n}a years}\label{fregions}
% 	\end{figure}
% Look at Liu ZHou paper for years

Although El Ni\~{n}o has been shown to have a significant impact on tropical storm genesis location, the annual storm numbers have been shown to lack an ENSO signal \citep{lander1994exploratory}.

%ENSO - TC numbers: Chan 1985 too? Under debate - see zhan 2012.  However, El Ni\~{n}o has shown to have reduction in numbers - see Lander paper

%Wang and Chan (2002) observed an increase in the number of TCs in the WNP during strong El Ni\~{n}o events, though no significant linear relationship between TC number and indices of ENSO. A reduction in the number of TCs occurring in the summer following and El Ni\~{n}o has been found, related to Walker circulation (Chan 1985).

%The canonical eastern Pacific El Ni\~{n}o and the central Pacific El Ni\~{n}o have been shown to have different effect on TC activity in the Western Pacific (ref).

%In a study of intense typhoons, it was found that the interannual variations in the WNP were caused largely by changes in the planetary-scale atmospheric circulation and thermodynamic structure associated with El Ni\~{n}o \citep{chan2007interannual}.


The Intertropical Convergence Zone, is the area encircling the earth near the equator where the northeast (N Hem) and southeast (S Hem) trade winds come together. Where the ITCZ is drawn into and merges with a monsoonal circulation, it is sometimes referred to as a monsoon trough. Most of the tropical cyclones that form in the WNP develop in the monsoon trough (MT) \citep{lander1994exploratory} and its location exhibits primary control on the distribution of TCs in the WNP \citep{wuinfluence}. 

\begin{figure}[h]
	\centering
	\noindent\includegraphics[width=20pc,angle=0]{MT_Lander.png}
	\caption{Long-term average of the low-level circulation during the summer in the Tropics of the western North Pacific. Bold zig-zag lines indicate ridge axes, and the bold dashed line indicates the axis of the monsoon trough. Arrows indicate wind direction. The locations of Guam (G) and Tokyo (T) are indicated. Source: \cite{lander1996specific}}\label{fig:MT}
\end{figure}

% when does the MT vary?
The monsoon trough is a climatological feature of low pressure and convergence (figure \ref{fig:MT}), with increased vorticity and it exhibits substantial migrations and changes of shape \citep{lander1996specific}. When the monsoon trough is defined as the contiguous region where long-term (1988-2010) mean July-November 850 hPa relative vorticity is positive, 73\% of all July-November tropical cyclones form within the monsoon trough \citep{molinari2013percentage}. This percentage displays interannual variation correlated with Ni\~{n}o 3.4 index, with more TCs forming in the MT in an El Ni\~{n}o phase \citep{molinari2013percentage}. The shift in genesis with ENSO phase has been related to the eastward extension of the monsoon trough and westerlies (associated with the increased cyclonic low-level vorticity) and the reduction in vertical wind shear near the date line \citep{lander1994exploratory, lander1996specific, wang2002strong}.
%See Camargo funny summary paper.
%Atmosphere - monsoon trough and subtropical ridge are important.See Harr and Elsbery 1991, 1995, Lander 1994, 1996, Liu and Chan 2002.

The Madden-Julian Oscillation (MJO) is a tropical mode of variability that has an intraseasonal time scale (30-90 days). It is characterised by an eastward progression of large regions of both enhanced and suppressed tropical rainfall, observed mainly over the Indian and Pacific Ocean. Over the western North Pacific (WNP), tropical cyclone activity appears to be strong when MJO related convection centre is in the WNP \citep{kim2008systematic}, however, no significant relationship with intensity has been found \citep{liebmann1994relationship}. As well as affecting the preferable region for genesis, TC tracks respond to the large-scale steering flows related to the MJO. When convection is in the equatorial Indian Ocean, tracks migrate eastward and when over the tropical WNP, they migrate westward \citep{kim2008systematic}.


%, Kim et al 1996.
%correct? See paper: Schreck et al (2012) 80-95\% WPAC TCs form in direct association with active MJO and/or active regions of various equatorial waves. 

%Studies have shown that the westerly phase of the Quasi Biennial Oscillation (QBO) corresponds to a larger number of TCs (refs) due to a decrease in the upper-tropospheric vertical shear over the tropics during the boreal summer. But does not hold during El Ni\~{n}o (See Chan 1995).

The subtropical high and mid-tropospheric flow pattern are important to TC movement \citep{chan1982tropical}. The strength and extend of the subtropical high influence where a TC will recurve (\ref{fig:STH}). %(ref steering section). How changeable is the sth?

\begin{figure}[h]
	\centering
	\noindent\includegraphics[width=16pc,angle=0]{typhoon16_1e.jpg}
	\caption{Subtropical High influence on TC movement. Source: }\label{fig:STH}
\end{figure}
%source http://www.weather.gov.hk/education/edu01met/01met_tropical_cyclones/ele_typhoon16_e.html

The Pacific Decadal Oscillation (PDO) exhibits variations in North Pacific SST over a decadal (20-30 years) timescale. It is similar to ENSO, with positive and negative SST anomalies, although it is on a much longer time scale and the most pronounced variations are at high latitudes, rather than in the Tropics  (figure \ref{fig:PDO}). The PDO Index is defined as the leading principal component of North Pacific monthly sea surface temperature variability (poleward of 20N for the 1900-93 period) (http://research.jisao.washington.edu/pdo/). The PDO has a significant impact on the subtropical high and mid-level steering during the peak TC season. It creates a north-south dipole of geopotential height over the WNP, affecting the subtropical high extension and intensity as well as the zonal winds \citep{liu2008interdecadal}.
%Check out Ho et al 2004 for PDO.
%NB wind stress changes with changing SST


\begin{figure}[h]
	\centering
	\noindent\includegraphics[width=20pc,angle=0]{pdo_pattern.png}
	\caption{Pacific Decadal Oscillation (PDO) positive phase. Figure from ADD SOURCE }\label{fig:PDO}
\end{figure}
%Add source to above: http://ds.data.jma.go.jp/tcc/tcc/products/elnino/decadal/pdo_doc.html


%WPSH is highly predictable and this can be used for tropical storm predictions \citep{wang2013subtropical}.

%Modes of variability - see Zhan 2012 review.
%Steering - Harr and Elsberry - straight, recurving, recurving north.
%IOD
%The intensity of a given storm depends on its surrounding environment.
%wang2013subtropical - Indian Ocean refs.

%Know about tropical waves - see Lu seasonal paper.

%IO importance - on MT?

Natural climate variability strongly modulates the seasonal statistics of tropical cyclones. 

%Affect of cyclone beforehand - upwelling is negative, but wave train is positive? (See camargo1)
%No - do not need to cover everything about TCs. Be specific and relevant.
%Something about historic activity can be divided into clusters, for example Camargo, Chan.

\subsubsection{Long-term variability}

Warming of the climate system is unequivocal \citep{stocker2013ipccb}, and as the internal energy available to weather systems such as tropical cyclones changes, activity of these phenomena is also expected to change. Warming of the ocean accounts for 93\% of the increase in the Earth's energy between 1971-2010, with 64\% of this in the upper ocean (0-700 m) \citep{rhein2013chapter}.

Modelling studies have consistently projected that greenhouse warming is likely to cause increases in the global average intensity of tropical cyclones and related rainfall rates \citep[e.g.][]{hill2011impact, knutson2010tropical, elsner2008increasing, bender2010modeled}, but some suggest that this is still within the range of natural variability. Many studies have shown that with an increase in SST in the coming decades, tropical cyclones will have increasing wind speeds \citep{bender2010modeled, murakami2012future, webster2005changes, emanuel2005increasing, knutson2010tropical, elsner2008increasing}, with \cite{emanuel2005increasing} suggesting that for every increase in SST of 1$^{\circ}$C, peak wind speed would increase by 5\%. The assumption is that the dominant effect of increasing carbon dioxide on tropical cyclones is through an increase in tropical mean sea surface temperature \citep{zhao2013robust}. However, modelling studies show that both spatial patterns of sea surface temperature warming and higher atmospheric carbon dioxide affect tropical cyclones independent of global sea surface temperature warming. % (ref)
% refs for within range of variability.
%The observed warming of the tropics of around 0.5$^0$C over the past 4 to 5 decades \citep{wang2010climate}. Consensus that the amount of energy will be the same, but higher lats will be increasingly exposed? \cite{huang2015change} suggested a suppressive effect of subsurface oceans on TC intensity in a warming environment due to sharpening of the subsurface vertical temperature profile, and therefore stronger ocean coupling (cooling).

Alongside SST warming, the effect of global sea level rise has consequences for damage caused by tropical cyclones. Global sea level has risen 10 to 20cm over last 100 years \citep{solomon2007climate} and it is very likely that the rate of global mean sea level rise during the 21st century will exceed the rate observed during 1971-2010 \citep{church2013sea}, so tropical cyclone-induced storm surge will become increasingly damaging phenomena. The water-holding capacity of air is a function of temperature, approximately doubling with each 10$^{\circ}$C increase in the range -20 to +45$^{\circ}$C \citep{fowler1995potential}. Therefore, in a warmer climate, precipitation from tropical cyclones will likely be more intense.

%That this rclationship is translated into pre¢ipitation potential is •vidcnccd by satdlitc obscrvations rcported by St•phcns (1990), which show a comparablc non-lincar rclationship b•tw••n s•a-surface tcmpcraturc and prccipitablc watet in a vcrtical air column ovcr the o¢cans. \citep{fowler1995potential}

%A number of factors that a warmer climate will affect on the activity and impact of tropical cyclones.

%Does it matter than here talking about modelling studies, before have talked about TCs in GCMs?

\subsection{Observations of tropical storms}
The historic record of tropical storm activity is longest for the North Atlantic, where records start in 1851 \citep{landsea2004atlantic}. Over the following decades, there has been a significant change in the methodology used to record such storms. In the early period, the main method for identifying tropical cyclones was by records of landfalling storms or by records of ships at sea \citep{vecchi2008estimates}. Further back in time, there were fewer ships and shipping lanes as well as fewer people living in the tropical and subtropical coastal regions \citep{landsea2007counting}, therefore it was possible that many storms were unrecorded. Aircraft reconnaissance in the West Pacific began near the end of the second World War and ended in 1987 \citep{knapp2013pressure}, and since then tropical storms have been monitored primarily by satellite \citep{lander1994exploratory}. The need for increased aircraft reconnaissance is acknowledged throughout the research and operational communities, with a recommendation from the World Meteorological Organisation (WMO) to 'realize the goal of regular and coordinated aircraft reconnaissance in the western North Pacific and other TC basins' \citep{wmoitwc8}. As part of government-led research, Japanese researchers are planning to have aircraft reconnaissance in the WNP from 2017 to at least 2020 \citep{nhkhurricanehunter}.

%, as many tropical cyclones(remove?)  were likely undetected over the tropical ocean in the pre-satellite era \citep{camargo2007cluster}

It is well established that although records begin many decades ago for some basins, they are only really trusted since the satellite era, ie. around the 1970s \citep{landsea2007counting} when global observations were possible. The observed time series of WNP tropical cyclone activity appears to be affected by artefacts of the changing observing system  \citep{lander1994exploratory, knapp2013pressure}, including the increasing use of satellite data and refining these methods, e.g. the Dvorak technique.
% and procedural changes . \citep{schreck2014impact} \citep{knapp2010international}

There are a number of agencies that produce Best Track records of tropical storm activity. The  International Best Track Archive Climate Stewardship (IBTrACS)\citep{knapp2010international} database consists of data from the World Meteorological Organization (WMO) Regional Specialized Meteorological Centres (RSMC) and Tropical Cyclone Warning Centres (TCWC) \citep{knapp2010international}. In the West Pacific region, data from the following agencies is available:

\begin{itemize}
	\item Fiji Meteorological Service (as RSMC Nadi)
	\item Japan Meteorological Agency (JMA) (as RSMC Tokyo)
	\item China Meteorological Administration’s Shanghai Typhoon Institute (CMA/STI)
	\item Hong Kong Observatory (HKO)
	\item U.S. Department of Defense Joint Typhoon Warning Center (JTWC)
\end{itemize}

For the WPAC region, JTWC storm data is most widely used \citep{knapp2010international}. All of the other agencies in this list provide data for a specific region in the West Pacific, rather than covering the entire basin. In some cases the same storms are tracked by multiple agencies, large differences in intensity can be found. There are also differences in lifetime as agencies employ different procedures for deciding when the first and last track point are determined.

%Modelling studies are vital, to learn more about TCs in the current climate and also to explore potential changes in the future.
% (tracks generallyu ok?), Also differences in lifetime - keep recording when goes ET or moves out of area of interest? Landfall in China goes further in CMA dataset (Matthias, pers comm). Different tracks, intensities, lifetime. eg. differences if calculating ACE. (see zhan2016cfs) Historical pressure record is more consistent between agencies than wind reports \citep{knapp2013pressure}.


\subsection{Representation of tropical storms in global climate models}

Current CMIP5 (Coupled Model Intercomparison Project 5) models assessed in the IPCC Fifth Assessment Report (AR5) have a finest horizontal resolution in the atmosphere of around 70km, but the average is about 200km \citep{climodaus}, an increase from the previous four versions of the report.
It is well established that the current generation of global climate models (GCMs) are of insufficient resolution to simulate tropical cyclones in detail due to their low resolution in comparison to the size of the storm core \citep{lin2012physically} and the complex nature of tropical cyclone structure, although enhanced computing capabilities and parametrisations have resulted in better representation of tropical cyclones \citep{zhao2009simulations, bengtsson2007may, walsh2007objectively}.

Modern GCMs are capable of producing structures that can be recognized as similar to tropical cyclones at resolutions as coarse as 100 km (Knutson et al. 2010), but without the details of the core \citep{mcdonald2005tropical}. There is a need for very high resolution in order to simulate realistic storm intensities, \citep{bender2010modeled, emanuel2008hurricanes} for example it has been suggested that 2 km or less is needed to represent important physical processes in the tropical cyclone eyewall \citep{gentry2010sensitivity}, with \cite{chen2007cblast} suggesting that a resolution of 1 km is required to resolve hurricane eye wall convection and wind maxima.
%  (not realistic for GCM? Talking about RCM?)


Resolution is important for simulating storm intensity, but less critical for simulating the annual number of tropical cyclones and their geographical distribution. \cite{zhao2009simulations} found that a model with a resolution in 20-100km range was able to simulate climatology and interannual variability of tropical storm frequency without realistic distribution of storm intensity. Similar results were found by \cite{strachan2013investigating}, with realistic interannual variability of storm occurrence requiring resolutions of 100 km or higher, but skill was basin dependent. Different models produce substantially different annual global tropical cyclone frequencies and geographical distributions due to model resolution and physical parametrisations \citep{zhao2013robust}. It has been found that there are larger differences in tropical cyclone distribution between models than in the same model at different resolutions, suggesting a greater influence of model specifics than resolution \citep{shaevitz2014characteristics}.
%60km, 25 km, 12 km (NWP size).

Roberts paper and my Reading paper - reanalysis.

%\subsection{Regional models}

%Regional models can be run at higher resolution than global models due to the reduced area over which calculations are being made. Typical resolution is ... and can be embedded in GCMs or GCMs downscaled.


%\newpage

%%%%%%%%%%%%%%%%%%%%%%%%%%%%%%%%%%%%%%%%%%%%%%%%%%%%%%%%%%%%%%%%%%%%%%%%%%%%


%How does my work compare to what has already been done.

\section{Extra-tropical cyclones} \label{ETC}
% Formation of TCs % Formation and structure of ETCs
% Tropical storm structure and energy lifecycle
% Influence of the ocean % Influence of the ocean
% Influence of the atmosphere % Influence of the atmosphere
% Tropical cyclone variability in the West Pacific % ETC activity
% Long-term variability % Long-term variability
% Observations % Observations
% Models % Models

Extra-tropical cyclones are a dominant feature of the mid-latitudes, associated with strong winds, precipitation and temperature changes.

%what kind of intensities? named?

\subsection {Formation and structure of extra-tropical cyclones}\label{ETC_form}
%Formation of jet streams? - no, not jet streams
%
%\begin{figure}[h]
%	\centering
%	\noindent\includegraphics[width=20pc,angle=0]{H:/Documents/Admin/ESA/jetstream3.jpg}
%	\caption{North hemisphere cross section showing jet streams and tropopause elevations. Source: \cite{noaa_jetstream}}\label{fig:jetstream}
%\end{figure}

An extra-tropical cyclone is a low pressure system that primarily gets its energy from a horizontal temperature gradient. They have frontal features, i.e. they are associated with cold fronts, warm fronts, and occluded fronts and in the northern hemisphere have winds counter clockwise. 
 
The first conceptualised model of the typical life-cycle of mid-latitude cyclones is the Norwegian model (e.g. Bjerknes and Solberg). This was proposed in the 1910s and 1920s and describes the evolution of a cyclone from an incipient frontal wave with cold and warm fronts, through intensification, maturity and decay (figure \ref {fig:norwegian_maps}).


\begin{figure}[h]
	\centering
	\noindent\includegraphics[width=16pc,angle=0]{cyclo_all.png}
	\caption{The Norwegian cyclone model in (1) map view (2) and 3-D view.  Source: \cite{norwegian}}\label{fig:norwegian_maps}
\end{figure}

%\begin{figure} % remove [h] and this appeared in the correct place
%	\subfloat[Initial state]{\includegraphics[width=1.8in]{cyclo1.png}}  % H:/Documents/Admin/ESA/Figures/
%	\subfloat{\includegraphics[width=1.8in]{wave23d.jpg}} 
%	\subfloat[Beginning stage]{\includegraphics[width=1.8in]{cyclo2.png}} 
%	\subfloat{\includegraphics[width=1.8in]{wave43d.jpg}}  		
%	\subfloat[Intensification]{\includegraphics[width=1.8in]{cyclo3.png}} 
%	\subfloat{\includegraphics[width=1.8in]{wave23d.jpg}} 
%	\subfloat[Mature stage]{\includegraphics[width=1.8in]{cyclo4.png}} 
%	\subfloat{\includegraphics[width=1.8in]{wave43d.jpg}} 
%	\subfloat[Dissipation]{\includegraphics[width=1.8in]{cyclo5.png}} 
%	\subfloat{\includegraphics[width=1.8in]{wave53d.jpg}} 
%	
%	\caption{The Norwegian cyclone model in (1) map view (2) and 3-D view.  Source: \cite{norwegian}}\label{fig:norwegian_maps}
%\end{figure}

One condition that favours cyclogenesis is a baroclinic zone, i.e. a region of large temperature change across a short horizontal distance near the surface, e.g. frontal zones. Above (near the tropopause) and parallel to this baroclinic zone is often a strong jet stream, driven by the thermal wind effect \ref{stull}. A wave develops on the front as an upper level low pressure system embedded in the jet stream moves over the front. As the air masses begin to rotate, defined cold and warm fronts develop. The wave intensifies and the low pressure centre deepens. The warm sector narrows as the cold front rotates around the cyclone faster than the warm front, and an occluded front develops where the cold front overtakes the warm front. As the cold front continues advancing on the warm front, the occlusion increases and eventually cuts off the supply of warm moist air, causing the low pressure system to gradually dissipate \citep{norwegian}. Precipitation occurs as air is forced to rise ahead of the warm front and the cold front. The more dense cold air undercuts the warm air and the less dense warm air rises above the cold air.

%http://weatherfaqs.org.uk/node/98
%As with the Norwegian cyclone model, the Shapiro-Keys model has an incipient cyclone develops cold and warm fronts, but in this case, the cold front moves roughly perpendicular to the warm front such that the fronts never meet, the so-called 'T-bone'. Also, a weakness appears along the poleward portion of the cold front near the low center, the so-called 'frontal fracture' and a back-bent front forms behind the low center. (In the final stage), colder air encircles warmer air near the low center, forming a warm seclusion. 

An alternative model is the Shapiro-Keyser model. The main difference from the Norwegian model is that the cold front moves roughly perpendicular to the warm front and they never meet. The occluded front is due to a weakness in the cold front near the low centre.
%The occluded front is an extension of the warm front rather than a result of the cold front catching up with the warm front. Both are valid and suited to different scenarios.. e.g....

%what about mositure?
%cyclogenesis involves sea-level pressure decrease as the low pressure centre deepens, upward-motion increase, and vorticity increase.

Energy from baroclinicity in the atmosphere. This is in contrast to tropical cyclones, where there is little temperature contrast across them and they get their energy from the underlying ocean. Much like tropical cyclones, extra-tropical cyclones are moved by the jet stream and by other large-scale components of the global circulation \citep{stull}.

Extra-tropical storms are driven by strong temperature gradients (section 3.1) and in the North Atlantic, often develop in baroclinic zone over Gulf Stream. In the MABL, the Gulf Stream directly influences air temperature and pressure fields but also has effects throughout the entire troposphere, with upper-tropospheric divergence exhibiting a structure similar to the surface convergence and precipitation patterns, all meandering with the Gulf Stream (\citep{minobe2008influence}).


In the marine atmospheric boundary layer (MABL), the Gulf Stream directly influences air temperature and pressure fields locally and also throughout entire troposphere, with upper-tropospheric divergence exhibiting a structure similar to the surface convergence and precipitation patterns, all meandering with the Gulf Stream (\cite{minobe2008influence}). Extra-tropical storms are driven by strong temperature gradients and in the North Atlantic, often develop in the baroclinic zone over Gulf Stream. Here, strong oceanic fronts that help to maintain the strong baroclinicity that is required to maintain them \citep{nakamura2004observed, nakamura2008importance, hoskins1990existence}

%Baroclinic instability is due to a horizontal temperature gradients in a rotating environment. 

%baroclinic wave, steering level. Steered by large scale, much like TCs?

Figure \ref{fig:ET_structure} shows a mature extra-tropical cyclone in more detail. An extra-tropical cyclone is typically larger than an average tropical storm, (approximately 1000 km)
As an extra-tropical cyclone moves eastwards, ahead of the warm front, there stratiform cloud as warm air is forced to rise and cool. Behind the warm front is the warm sector, of relatively warm air and generally clear skies. At the cold front, dense cool air undercuts the warm, moist air and forces it steeply upwards, with a band of cumuliform clouds, heavy rain and thunderstorms. Following the passage of the cold front, there is cooler air, bright skies and showers, and a marked veer in wind direction. The near surface winds converge towards the low pressure centre. Structurally, extra-tropical cyclones are 'cold-core', unlike tropical cyclones, which are 'warm core'.
%Cold sector surface flow is from the northwest
%Warm sector surface flow is from the southwest, bringing warm low-latidude air poleward and upward 'warm conveyor belt', associated with weak turbulent heat fluxes at the surface (warm path pahper?). Warm conveyor belt, cold conveyor belt, dry intrusion.

\begin{figure}[h]
	\centering
	\noindent\includegraphics[width=20pc,angle=0]{ET_structure.png}
	\caption{Components of a typical extra-tropical cyclone in the N. Hemisphere. Light grey shows clouds, dark grey arrows are near-surface winds, thin black lines are isobars (kPa), thick black lines are fronts, and the double-shaft arrow shows movement of the low centre 'L'. Source: \cite{stull} }\label{fig:ET_structure}
\end{figure}



\subsection{Influence of the ocean on extra-tropical cyclones} \label{etc_ocean}

%Visbeck chapter: "However, outside of the deep tropics, most studies find that the ocean largely reacts to the high frequency changes of the atmospheric forcing and that its influence back to the atmosphere is weak on time scales shorter than a decade."

Extratropical ocean-atmosphere interaction dominated by atmosphere forcing the ocean, but with variability with oceanic processes more important to SST in the vicinity of WBCs (smirnow).

Major storm tracks are organised along or just downstream of the main oceanic frontal zones (Nakamura et al 2004).

\cite{nakamura2008importance} and booth et al 2010 - air-sea heat exchanges at oceanic fronts restore the baroclinicity of the atmosphere at low levels.

Atmosphere-ocean interactions are their strongest over WBCs, e.g Gulf Stream. Strong fluxes of heat and moisture anchor the storm tracks to the WBCs (Nakamura et al 2004). allows for recurrent development of storm tracks - creates baroclinicity?

Can the ocean affect intensification/dissipation?

%\cite{nakamura2004observed}
%As the surface air temperature over the open ocean is linked to SST underneath, maritime surface baroclinic zones tend to be anchored along oceanic fontal zones [NS04]. Though acting as thermal damping for the evolution of individual  eddies, heat exchange with the underlying ocean, on longer time scales, can act to restore atmospheric near-surface baroclinicity against the relaxing effect by atmospheric eddy heat transport, as evident in sharp meridional contrasts in upward turbulent heat fluxes observed climatologically across midlatitude frontal zones [Oberhuber, 1988]. Some observations are shown in section 2 to suggest that SST anomalies in a midlatitude frontal zone can likely play a more active role in the air–sea interaction than act to damp  tmospheric anomalies thermally Hoskins and Valdes (1990) mean diabatic heatin as a result of the warm ocean current restores the meridional atmostpheric temperature gradient and therfore baroclinciiy, that is vital to the storm tracks existence.

\subsection{Influence of the atmosphere on extra-tropical cyclones} \label{etc_atm}

A baroclinic atmosphere is required for cyclogenesis and there is often an associated a strong jet stream running parallel at upper levels. Genesis typically along the polar front, separation between warm subtropical air and cold polar air.

Jet streams are important source regions for storm development (Holton), developing in the jet entrance, being advected downstream and then decaying in the jet exit region (Holton).

Can the atmosphere affect intensification/dissipation?

Form where surface temperature gradients are large and the jet stream influences their speed and direction of travel. 
%Different types of instability - baroclinic, barotropic etc. Chang paper


\subsection {Extra-tropical cyclone variability in the North Atlantic} \label{etc_var} %ocean, atmos, variability

at their maximum when surface baroclinicity is strongest, ie during winter.

Extra-tropical storms are driven by strong temperature gradients and in the North Atlantic, often develop in the baroclinic zone over Gulf Stream. Here, strong oceanic fronts that help to maintain the strong baroclinicity that is required to maintain them \citep{nakamura2004observed, nakamura2008importance, hoskins1990existence}. In the MABL, the Gulf Stream directly influences air temperature and pressure fields but also has effects throughout the entire troposphere, with upper-tropospheric divergence exhibiting a structure similar to the surface convergence and precipitation patterns, all meandering with the Gulf Stream (\citep{minobe2008influence}). Many studies have examined air-sea interactions associated with western boundary currents  \citep{ma2015distant, shaman2010air, kwon2010role, czaja2001observations, czaja2002observed, minobe2008influence, smirnov2015investigating, kelly2010western, cayan1992latent}, and it has been shown that the Gulf Stream plays an important role in the climate of the northern hemisphere, mainly through affecting the eddy driven jet and associated extra-tropical cyclones \citep{sampe2010significance, nakamura2008importance, booth2012sensitivity, small2014storm, woollings2012response, vanniere2017contribution, vanniere2017cold}. 

\begin{figure}[h]
	\centering
	\noindent\includegraphics[width=44pc,angle=0]{NAO_all_MO.png}
	\caption{The North Atlantic Oscillation (NAO) negative and positive phases. Source: Met Office }\label{fig:NAO}
\end{figure}
% https://www.metoffice.gov.uk/learning/atmosphere/north-atlantic-oscillation

In the North Atlantic, the leading mode of winter climate variability is the North Atlantic Oscillation (NAO), which is related to sea surface temperature (SST) and storm track activity \citep{vallis2008local}. The NAO is the large-scale change in pressure between the subpolar Icelandic Low and subtropical Azores High. This pressure difference determines the strength of the westerly winds blowing across the North Atlantic. The NAO negative phase occurs when this pressure difference is small and the NAO positive phase occurs when this is large (figure ref{fig:NAO}). The positive phase is associated with a strong zonal jet stream, bringing warm and wet conditions to north-west Europe. In the negative phase, the jet stream is displaced and atmospheric conditions are 'blocked', ie high pressure, stable weather, instead of transient eddies, bringing cold and dry weather to north-west Europe. Changes in the phase of the NAO are associated with basin-wide changes in the intensity and location of the North Atlantic jet stream and storm track. More frequent occurrence of extreme cyclones in positive NAO phase because a larger area suitable for growth \citet{pinto2009factors}.

The NAO also affects the ocean. The leading pattern of SST variability in boreal winter shows a tripole structure, with cold anomalies in the subpolar and subtropical regions, and a warm anomaly in the mid latitudes centred off Cape Hatteras (figure \ref{fig:NAO_SST}). MAKE MY OWN PLOT - EOF SHOWING LEADING PATTERN. This SST structure is driven by air-sea heat exchanges and surface induces Ekman currents associated with NAO variations (EKMAN, so not geostrophic?) (Marshall et al 2001, Visbeck et al 2003)(see Hurrell)

\begin{figure}[h]
	\centering
	\noindent\includegraphics[width=20pc,angle=0]{czaja_NAH.png}
	\caption{North Atlantic Oscillation (NAO) SST tripole. Source: \citet{czaja2002observed}. COULD REMAKE WITH MY DATA - EOF OF SSTS in DATASET}\label{fig:NAO_SST}
\end{figure}

%There is not only one picture of the pressure pattern associated with the NAO, as....blocking \citep{woollings2018daily}

North Atlantic eddy-driven jet ( as opposed to thermally driven subtropical jet).

Marshall 2001

Cyclones play a role in steering NAO phase (Pinto, Feldstein, Benedict, Franzke)

wave breaking, Rockies, storm track, large scale, wave packets, blocking episode time scales and filtering
relate NAO to weather scales, ie blocking and therefore storm steering/presence
NAO from wave breaking events?

Blocking is quasi-stationary persistent high pressure system. \citet{shabbar2001relationship} showed that blocking is strongly related to the phase of the NAO, with 75\% more blocking days in winter NAO- than during NAO+, with on average longer blocking episodes in NAO-, showing about 30\% of the variation in wintertime blocking in the North Atlantic is accounted for by the NAO. Blocking on storms...
It is well established that many forecast busts over Europe are due to poor representation of atmospheric blocking \cite{rodwell2013characteristics}.
\citep{scaife2011improved} Improved blocking after storm track stuff
\citep{dunstone2016skilful} improved forecasts of winter blocking

NAO time series? Difference with PC or station

\begin{figure}[h]
	\centering
	\noindent\includegraphics[width=24pc,angle=0]{NAO_TS_NOAA.png}
	\caption{North Atlantic Oscillation (NAO) time series. Source: NOAA }\label{fig:NAO_TS_NOAA}
\end{figure}
%https://www.ncdc.noaa.gov/teleconnections/nao/

%Gulf Stream section? Or just leave for chapter intro??

%Sheldon(The atmosphere above the Northern Hemisphere’s western boundary currents (the Gulf Stream and Kuroshio) are maximums in the winter atmospheric variability on a synoptic timescale of 2-6 days (Lau and Wallace, 1979; Blackmon, 1976; Hoskins and Valdes, 1990). These regions of synoptic variability are called the storm tracks and the variability is measuring the growth of extra-tropical cyclones that occur there (Dacre and Gray, 2009).)

%shear instability not parametrised in current generation of GCMs.

%Explain in depth PV and equivalent potential temperature

In \citet{pinto2009factors} future climate total North Atlantic cyclones reduced in number by 10\% apart from near the British Isles, with an increase in track density and extremes due to intensified jet stream close to Europe. \citet{zappa2013multimodel} found in DJF, an increase in cyclone frequency in central Europe and a decrease in the Norwegian and Mediterranean Seas, with the total number of cyclones decreases in both DJF (\textasciitilde{4}\%). "A slight basin-wide reduction in the number of cyclones associated with strong winds, but an increase in those associated with strong precipitation. However, in DJF, a slight increase in the number and intensity of cyclones associated with strong wind speeds is found over the United Kingdom and central Europe."


%%%%%%%%%%%%%%%%%%%%%%%%%%%%%%%%%%%%%%%%%%%%%%%%%%%%%%%%%%%%%%%%%%%%%%%%%%%%%%%%%%%%%%%%

\subsection{Atmospheric instability and convection} % Move to chapter intro??

\subsubsection {Gravitational instability and upright convection}

In a hydrostatically balanced atmosphere, the mean-state potential temperature ($\theta$) decreases with height. If the potential temperature cools with height, there will be overturning as the cool air sinks. This is gravitational instability (\ref{eqgravinst}), which releases upright convection and is purely vertical.

\begin{equation} \label{eqgravinst}
\frac{d\theta}{dz} < 0
\end{equation}


\subsubsection {Inertial instability and horizontal movement}
Inertial instability operates in the horizontal plane and is detected by absolute momentum (M). Absolute momentum, as defined by \cite{eliassen1962vertical} is the rotation of the Earth plus the rotation due to variations in the wind field:

\begin{equation} \label{eqM}
M = V + fx
\end{equation}


%\begin{equation} \label{eqN}
%N = U - fy
%\end{equation}

where V is the frontal velocity and x is the distance along the transverse plane to the front. When the absolute momentum decreases in the x-direction then the atmosphere is unstable:

\begin{equation} \label{eq_iner_inst}
\frac{dM}{dx} < 0
\end{equation}


\subsubsection {Shear (symmetric) instability and slantwise convection}

Shear (or symmetric) instability is a combination of gravitational and inertial instability. A parcel may be inertially stable to horizontal displacements and gravitationally stable to vertical displacements, but may be unstable to slantwise displacements by shear instability. 

A barotropic atmosphere is one where the density depends solely on pressure, whereas in a baroclinic atmosphere the density depends on both the temperature and the pressure. In a barotropic flow, potential temperature surfaces (isentropes) are horizontal and absolute momentum surfaces (M) are vertical. In a baroclinic environment, the isentropes tilt upwards on the cold side and downwards on the warm side. The absolute momentum surfaces slope in response to the vertical shear from thermal wind balance, tilting up on the cold side and down on the warm side.
%
%\begin{figure}[h]	
%	\includegraphics[width=34pc,angle=0]{H:/Documents/Thesis/phd-thesis-template-2.2.2_AC/phd-thesis-template-2.2.2/Figs/barotropic2_new.pdf}
%	\caption{Barotropic and baroclinic environments in the horizontal and vertical, and resultant pressure gradient force (PGF) and thermal wind (cross in circle). Thin black lines are isentropes.}\label{fig:barotropic}
%	\centering
%\end{figure} 

The  M-$\theta$e relationship states that if the $\theta$e surfaces are steeper than the M surfaces, there is symmetric instability. Any slantwise displacement occurring between the slopes of these surfaces will release the symmetric instability and the parcel will be accelerate in the direction away from the original position. Figure \ref{fig:symm_inst} illustrates the M-$\theta$e relationship, with a symmetrically unstable atmosphere on the left and a symmetrically stable atmosphere on the right. Only moist slantwise instability occurs in the Earth's atmosphere \cite{bennetts1979conditional} and so $\theta$e is used. This instability has a 2D assumption that there is no variation in the along front direction. 


\begin{figure}
	\centering	
	\includegraphics[width=26pc,angle=0]{mocrette_diagram2_screen.png}
	\caption{Shear (symmetric instability). Isentropes are shown in red dashed lines and lines of constant absolute momentum (M) in solid green. Thickness of the lines increases towards  higher values. Both environments are baroclinic and $\theta$e and M are low in the bottom left and high in the top right. The black arrows show direction of movement of air parcels. In both panels a and b, if a parcel is moved to the left, it is inertially stable as it is moving towards an area of reduced absolute momentum. In both panels a and b, if a parcel is moved vertically, they are gravitationally stable as potential temperature is increasing in this direction and so the parcel will return to it's original position. However, in panel a, if an air parcel is moved along arrow x, it is moving towards lower potential temperature and high M, so is symmetrically unstable. In panel b, it will move towards higher $\theta$e and low M, and is symmetrically stable. Modified from \cite{morcrette2004radar}}\label{fig:symm_inst}
\end{figure}


%\begin{figure}
%	\centering	
%	\includegraphics[width=26pc,angle=0]{H:/Documents/Thesis/phd-thesis-template-2.2.2_AC/phd-thesis-template-2.2.2/Figs/morcrette_cx.png}
%	\caption{1.6 The distribution of saturated equivalent potential temperature (red lines) and
%		geostrophic absolute momentum Mg (blue lines) in the broad region around a cold front. Far
%		from the front, the atmosphere is barotropic, within the cold frontal region the atmosphere is
%		increasingly baroclinic, very close to the front, a conditionally symmetrically unstable region
%		may be present. In this example, there is also a small region of conditional instability below
%		the symmetrically unstable region.Source: \cite{morcrette2004radar}}\label{fig:symm_inst2}
%	
%\end{figure}

It has been suggested that slantwise convection from the release of shear instability can explain banded precipitation often seen at cold fronts \citep{bennetts1979conditional, seltzer1985possible}, caused by cells of alternating rotation creating updrafts and downdrafts.

The shear instability diagnostic that is used in this study is diagnosed using:
% dubar/dp x (u'w')bar + dvbar/d[ x (v'w')bar
\begin{equation} \label{eq_diag}
-\overline{w'u'} . \frac{\partial{\overline u}}{\partial z} + \overline{w'v'} . \frac{\partial{\overline v}}{\partial z}
\end{equation}

\begin{equation} \label{eq_diag}
\frac{\partial}{\partial{T}} EKE = -\overline{w'u'} . \frac{\partial{\overline u}}{\partial z} + \overline{w'v'} . \frac{\partial{\overline v}}{\partial z} + ...
\end{equation}

The instability in the x (u) and y (v) directions are combined. The covariance of small-scale vertical motions ($\omega$') and horizontal vertical motions (u' or v') is calculated and then the  product with the vertical shear of low pass winds ($\overline{u}$ or $\overline{v}$)  is created. This is a momentum flux that is purely mechanical and has no dependence on temperature.

%Energy cascade. Also small scale feeds back on the large scale - the diagnostic is the bar value.
%
%Mechanical only.... not like capes????
%Thermodynamic and dynamic mechanism???? what is all this about?
%Shear diagnostic is just vertical and horizontal winds.

%Holton P 279 - symmetric baroclinic instability
%Absolute momentum and potential temperature conservation
%
%WHAT IS MOMENTUM AND ENERGY
%
%\begin{equation} \label{eq_EKE}
%\frac{\overline{(u'^{2}+v'^{2})}}{2} 
%\end{equation}
%EKE in relation to diagnostic?


%Buoyancy (b). g is acceleration due to gravity \(9.8ms^{-1}\). {$\rho$} is density.
%
%\begin{equation} \label{eq_b}
%b = -g\frac{\rho}{\rho_0}
%\end{equation}
%
%Also calculated upright buoyancy (w'{$\alpha$})
%
%Relate to buoyancy and EKE

\subsection {Observations of extra-tropical cyclones}

Started naming ETCs a few years ago in Europe.



\subsection {Representation of extra-tropical cyclones in global climate models}

%Different results - different identification and tracking, intensity measures \citep{ulbrich2009extra}.
GCMs generally simulate the storm tracks well (d'Andrea et al 1998)as they are large-scale phenomenon of the atmospheric circulation. Also climate models can capture the structure of ETCs \citep{catto2010can}.

Climate models (HiGEM (0.83x1.25) and ERA-40) can capture the structural features of extra-tropical cyclones \citep{catto2010can} analysing warm conveyor belt, cold conveyor belt and dry intrusion.

Current generation weather and climate models are too coarse to properly represent all of the features within in extra-tropical cyclone, and such features are routinely parametrised. Convective instability is included in these schemes and arises from thermal differentials within the atmospheric column. Shear instability is also present, but is not parametrised nor resolved.

Reanalyses with higher resolution had more intense ETCs (like with TCs). Hodges et al 2011 Wang et al 2016 (see Rohrer paper). 
GCMs produce realistic ETC precipitation on average \citet{booth2018evaluation}.